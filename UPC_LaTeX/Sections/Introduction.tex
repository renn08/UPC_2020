\section{Introduction}
Quadcopter unmanned aerial vehicles (UAVs) are used widely both in industry and in academic research nowadays. Some applications with precise operation requires high stability of the platform.

Wind is the major disturbance source effecting the stability of the UAVs. The real wind usually has time-dependent magnitude and direction, which can easily threaten the safety of the equipment. Thus the maximum wind speed that allows safe operation is an important parameter to estimate the UAV performance.

In this article, we are going to develop a simulation method to determine the maximum wind speed that can allow the drone to stay within $20$ cm of the target. 

%Sketch the method you are going to use. 
%Give an overview (one sentence) of the structure of your paper.
In Model section, the physics simulation system is first proposed, which models the dynamic of the drone given the wind. Then is the thrust controller, which generates the best response given the wind and the drone's current info. Finally, the wind simulator is  proposed, which can both generate periodic or noised wind. 

% To present the results of the simulation method, we also developed a simple flight control system and connect it with the simulation program. In real applications, the simulation can be applied to some more advanced control system and test the performance of the quadcopter. 

In Results section, we will first give the brief summary of our model's simulation result and give the maximum wind speed that allows safe operation. Then, the detailed data will be given with clear image illustration to validate our statement.

Finally,we will draw conclusions and discuss the advantages and  and advantages of our model and give comments on how to improve the model.
 



% Roller coaster is one of the most exciting recreation facilities in amusement park. In this article, we are going to design a roller coaster which is safe and exciting. We will first give a concept diagram of our roller coaster’s whole trajectory, then divide it into five parts and analyse them respectively. We are going to use the basic kinematics laws to construct second order ODEs for the motion of the roller coaster car, and apply the Euler Method to obtain an approximate solution for the ODEs, from which we could obtain the velocity and acceleration at any instant of time.
% In Model Section, we will first give our method of judgement of safety and excitement, then introduce the basic laws and methods we will use in this project. Moreover, for each part of our trajectory, we will give a simple model with parameters.
% In Result Section, we will first give the overall results of our whole model, and give the proof of the safety and excitement. Then we will explain in details about how we obtain the results and the motion of the car in each part of the trajectory.
% Finally we will draw a conclusion and discuss the limitations and advantages of our model and give some suggestions about how to improve the model.
