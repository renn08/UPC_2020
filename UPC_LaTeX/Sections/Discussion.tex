\section{Discussion}
\subsection{Conclusion}
In this paper, we proposed a physics simulation system to simulate
\begin{itemize}
    \item the dynamics of the quadcopter;
    \item a thrust control system which controls the thrusts of the quadcopter, keeping the quadcopter near the origin;
    \item a wind simulator to simulate winds that vary in time and magnitude through time.
\end{itemize}

These three parts are combined to find the maximum wind speed, which is
\begin{itemize}
    \item for dominant wind with noise,
    \item for periodic wind.
\end{itemize}

We find that the maximum windspeed allowed for the quadcopter to stay within the desired range for at least $30s$ when the dominant wind with noise was applied is $5m/s$.

For the periodic wind pattern, our control model cannot handle wind with varying magnitude as perfectly as it does with dominant wind patterns. With fixed amplitude, the stability becomes worse at the frequency of the change of wind's velocity increases. 

Under frequency $f = 0.01$ Hz, the performance reaches our expectation only when the amplitude of the velocity of wind is no larger than $0.95 m/s$.

We can accordingly conclude that periodic wind poses greater threats to safety operation than the dominant wind with noise. 

\subsection{Advantages}
Though it's a simplified model, our simulation has many advantages.
\begin{enumerate}
    \item The physics simulation system is accurate.
    
    As the rigid body dynamics theorem is used comprehensively, our physics simulation system will have the exact behavior as the according simplified model in reality.
    \item The thrust control algorithm is original and has strong physics interpretation.
    
    The thrust control algorithm was derived purely by solving the Newton-Euler equations and extracting terms with linear dependence on the thrust from the  accelerations (both translational and angular). Thus, compared to the standard PID control algorithm \cite{bib4}, it does not only have far more less parameters to adjust, but also a much clearer connection to the model it controls.
    % TODO: ref a PID paper
    
    \item The wind simulator simulates different types of winds and is adjustable.
    
    Through the simulation mathematical models of the three parts of wind speed that compose the wind field, dominant wind, periodic wind and random wind are used to simulate the change process of wind speed over a period of time, so that the randomness, intermittentness and the mutant characteristics of the real wind field under external interference can be considered. 
    
\end{enumerate}

\subsection{Limitation and possible refinement}
    Due to time limit, our model still has some limitations that need further refinement.
\begin{enumerate}
    \item The shape of the quadcopter.
    
    In this article, we simplify the major part of the quadcopter as a sphere and omit the structure connecting the rotors and the middle platform. However, actually the moment of inertia of the rod structures also matters in the equations of rigid body dynamics. However, in our model, due to the symmetry, this additional item can be replaced by changing the mass distribution of our model and will not effect the final results.
    
    \item The omitted Gyro force 
    
    In our thrust control system, we neglect the Gyro force when the angular speed is changing according to time. However, the neglect is reasonable because the magnitude of the force is 1-2 order smaller than other force, e.g. lift force or drag force acting on the middle sphere. Also, this force is considered in our physics simulation to accurately predict the motion of the quadcopter.
    
    \item Air drag
    To reduce the complexity of calculation, we omitted the drag force acting on four rotors. According to the study on the aerodynamics of the propellers\cite{bib1}, the drag force mainly depends on the translational velocity of the quadcopter and the angular speed of pitch, roll and yaw. The simulation can be improved by adding these items into our functions of force in the programs.
    
\end{enumerate}

